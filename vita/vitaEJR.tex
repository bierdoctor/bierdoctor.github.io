
% \documentclass[10pt]{report}
\documentclass[9pt]{extarticle}

% specify packages to include
\usepackage[margin=1.5in,top=1in,nohead]{geometry}	% package for defining aspects of page dimensions

\geometry{letterpaper}			% ... or a4paper or a5paper or ... (define paper size)
\usepackage[parfill]{parskip}	% begin paragraphs with an empty line rather than an indent
\usepackage{lastpage}			% allows creation of references to last page of document (page 21 of 47) via LastPage marker
\usepackage{graphics}			% allows importing graphics in eps format
\usepackage{rotating}			% allows rotation of in-line figures, tables and captions
\usepackage{xspace}			    % adds space at the end of a macro designed for use in text,
\usepackage{hyperref}			% create hyperlinks within the document; only works with pdflatex
\usepackage{array}				% supports more complex column formatting
\usepackage{xcolor}			    % driver-independent specification of colors
\usepackage{tipa}				% phonetic symbols (additional special characters)

\usepackage{fancyhdr}			% supports more complex headers and footers
\usepackage[round]{natbib}		% bibliography creation
\usepackage{newcent}			% default serif font
\usepackage{avant}				% default sans serif font
\usepackage[symbol]{footmisc}	% footnotes with symbols
%\usepackage{utopia}			% serif
%\usepackage{helvet}			% sans serif

% redefine format of \section command
\makeatletter
\renewcommand{\section}{%
  \@startsection{section}{1}{0em}{\baselineskip}{3pt}{\large\bfseries\textsc}}
\makeatother
\setcounter{secnumdepth}{0}

% define aspects of page setup
\pdfpagewidth 8.75in
\pdfpageheight 11in 
\setlength\textheight{8.5in}
\widowpenalty=10000
\raggedright

% define the absence of header, and footer for each page
\renewcommand{\headrulewidth}{0pt}
\pagestyle{fancy}
\fancyhead{}
\fancyfoot{}
\fancyfoot[LO,LE]{\emph{\small Dr. Emilee Rader} \\ \emph{\small updated: \today}}
\fancyfoot[RO,RE]{\small \thepage}
\pagenumbering{arabic}	% vs. roman numerals

% make a tilde that looks ok
\newcommand{\mytilde}{\raise.17ex\hbox{$\scriptstyle\mathtt{\sim}$}}

% footnotes with symbols
% \renewcommand{\thefootnote}{\fnsymbol{footnote}} 

%----------------------------------------------------------------------
% TITLE
\begin{document}
{\LARGE{\textbf{Emilee Rader}}}  \\ \vspace{3pt}
\textbf{Curriculum Vitae}


%----------------------------------------------------------------------
% CONTACT INFO
\begin{tabbing}
Communication Arts \& Sciences \hspace{1.5in}\=  \\
404 Wilson Road, Room 409 \\
East Lansing, MI 48824-1212 \\
\href{mailto:emilee@msu.edu}{emilee@msu.edu} \\
\href{http://msu.edu/~emilee}{http://msu.edu/\mytilde emilee}  \\
\href{http://bitlab.cas.msu.edu/}{http://bitlab.cas.msu.edu/}  \\\\
Office: 517-432-1334  \textpipe  ~Fax: 517-355-1292  \textpipe  ~Mobile: 847-514-0481 \\\\\\
\end{tabbing}



%----------------------------------------------------------------------
% APPOINTMENTS
\section{Appointments}
\begin{tabbing}
2017-  \hspace{0.435in}\= \emph{Associate Professor and AT\&T Scholar} \\
\> \href{http://tc.msu.edu/}{Department of Media and Information}, College of \\
\> Communication Arts \& Sciences, Michigan State University \\\\
2011-17  \hspace{0.3in}\= \emph{Assistant Professor and AT\&T Scholar} \\
\> \href{http://tc.msu.edu/}{Department of Media and Information}, College of \\
\> Communication Arts \& Sciences, Michigan State University \\\\
2010-11  \hspace{0.3in}\= \emph{Assistant Professor} \\
\> \href{http://www.communication.northwestern.edu/departments/communicationstudies/}{Department of Communication Studies}, School of \\
\> Communication, Northwestern University \\\\
2009-10  \hspace{0.3in}\= \emph{Computing Innovation Fellow} \\
\> Center for Technology and Social Behavior, Northwestern University \\\\
1999-04 \> \emph{Senior Human Factors Scientist} \\
\> User Centered Solutions Lab, Motorola Labs 
\end{tabbing}


%----------------------------------------------------------------------
% EDUCATION
\section{Education}

\begin{tabbing}
\emph{Ph.D., Information (Fall 2009)} \\
\href{http://www.si.umich.edu/}{School of Information}, University of Michigan \\\\
\emph{Master of HCI (August 1999)} \\
\href{http://www.hcii.cmu.edu/}{Human Computer Interaction Institute}, Carnegie Mellon University \\\\
\emph{B.S., \href{http://psych.wisc.edu/}{Psychology} (Fall 1995)} \\
University of Wisconsin at Madison (Graduated with Distinction) 
\end{tabbing}


%----------------------------------------------------------------------
% FUNDING
\section{Grants and Fellowships}

\begin{tabbing}
\textbf{Title / Source} \hspace{2.75in}\= \textbf{Period} \hspace{0.5in}\= \textbf{Amount} \\
\emph{TWC: Small: Designing a Coordination Mechanism} \> 2015-2020 \> \$463,107 \\
\emph{for Managing Privacy as a Common-Pool Resource} \\ 
National Science Foundation (\href{http://www.nsf.gov/awardsearch/showAward?AWD_ID=1524296}{CNS-1524296}) \\ 
Role: PI \\\\\\\\\\ % newpage here

\textbf{Title / Source} \hspace{2.75in}\= \textbf{Period} \hspace{0.5in}\= \textbf{Amount} \\
\emph{Workshop on Trustworthy Algorithmic Decision-Making} \> 2017-2018 \> \$93,909 \\
National Science Foundation (\href{https://www.nsf.gov/awardsearch/showAward?AWD_ID=1748381}{CNS-1748381}) \\ 
Role: PI \\\\

\emph{III: HCC: Small: Effects of Automated Information} \> 2012-2017 \> \$486,093 \\
\emph{Selection and Presentation in Online Information Systems} \\ 
National Science Foundation (\href{http://nsf.gov/awardsearch/showAward.do?AwardNumber=1217212}{IIS-1217212}) \\ 
Role: PI \\
	\hspace{1cm}REU supplement 2013: \$16,000 \\\\

\emph{TC: Small: Collaborative Research: Influencing} \> 2011-2017 \> \$241,589 \\
\emph{Mental Models of Security} (joint with Dr. Rick Wash) \\ 
National Science Foundation (\href{http://nsf.gov/awardsearch/showAward.do?AwardNumber=1115926}{CNS-1115926}) \\ Role: PI \\
	\hspace{1cm}REU Supplement 2015: \$8,000 \\
	\hspace{1cm}REU Supplement 2014: \$8,000 \\
	\hspace{1cm}REU Supplement 2013: \$8,000 \\
	\hspace{1cm}REU Supplement 2012: \$16,000 \\\\

\emph{\href{http://cra.org/ccc/leadership-development/cifellows/}{Computing Innovation Fellowship}}  \> 2009-2010 \> \$140,000 \\ 
Computing Research Association (CRA) and NSF \\ 
(in conjunction with \href{http://nsf.gov/awardsearch/showAward?AWD_ID=0937060}{CNS-0937060})
\end{tabbing}


%----------------------------------------------------------------------
% PUBLICATIONS
\section{Peer-Reviewed Publications}

\vspace{-5pt}
\footnotesize
Except where noted by \footnote{Data analysis and manuscript preparation only.\label{rolefn}}, Dr. Rader made substantial contributions to each publication in all of the following areas: 1) project administration, including securing funding and resources; 2) research conception, design, and development; 3) data collection; 4) data analysis and interpretation; and 5) manuscript drafting and preparation. Undergraduate student coauthors are denoted by \footnote{Undergraduate student coauthor.\label{ugfn}} and graduate student coauthors are denoted by \footnote{Graduate student coauthor.\label{gdfn}}.\\
\normalsize
\vspace{5pt}

Dev, J.\footref{gdfn}, \textbf{Rader, E.} and Patil, S. (2020). Why Johnny Can't Unsubscribe: Barriers to Stopping Unwanted Emails. To appear in \emph{CHI 2020}. Acceptance rate 24\%. doi: 10.1145/3313831.3376165

\textbf{Rader, E.} and Munasinghe, A.\footref{ugfn} (2019). ``Wait, Do I Know This Person?'': Managing Misdirected Email.  \emph{CHI 2019}. Acceptance rate 24\%. doi: 10.1145/3290605.3300520

Yang, Y., \textbf{Rader, E.\footref{rolefn}} Peters-Carr, M., Bent, R., Smilowitz, J., Guillemin, K., and Rader, B. (2019). Ontogeny of alkaline phosphatase activity in infant intestines and breast milk. To appear in \emph{BMC Pediatrics}. doi: 10.1186/s12887-018-1379-1

\textbf{Rader, E.} Cotter, K.\footref{gdfn} and Cho, J.\footref{gdfn} (2018). Explanations as Mechanisms for Supporting Algorithmic Transparency.  \emph{CHI 2018}. Acceptance rate 25\%. doi: 10.1145/3173574.3173677

\textbf{Rader, E.} and Slaker, J.\footref{gdfn} (2017). The Importance of Visibility for Folk Theories of Sensor Data. \emph{SOUPS 2017}. Acceptance rate 27\%. 

Wash, R., \textbf{Rader, E.}, and Fennell, C.\footref{gdfn} (2017). Can People Self-Report Security Accurately? Agreement Between Self-Report and Behavioral Measures. \emph{CHI 2017}. Acceptance rate 25\%. doi: 10.1145/3025453.3025911

\textbf{Rader, E.} (2017). Examining User Surprise as a Symptom of Algorithmic Filtering. \emph{International Journal of Human Computer Studies}, 98, 72-88, doi: 10.1016/j.ijhcs.2016.10.005

Wash, R., \textbf{Rader, E.}, Berman, R.\footref{ugfn}, and Wellmer, Z.\footref{ugfn} (2016) Understanding Password Choices: How Frequently Entered Passwords are Re-used Across Websites. \emph{SOUPS 2016}. Acceptance rate 28\%.

Jung, Y.\footref{gdfn} and \textbf{Rader, E.\footref{rolefn}}. (2016). The imagined audience and privacy concern on Facebook: differences between producers and consumers. \emph{Social Media $+$ Society}, 2(2), doi: 10.1177/2056305116644615.

\textbf{Rader, E.} and Wash, R. (2015). Identifying Patterns in Informal Sources of Security Information. \emph{Journal of Cybersecurity}, 1(1), 121-144, doi: 10.1093/cybsec/tyv008.

Wash, R. and \textbf{Rader, E.} (2015). Too Much Knowledge? Security Beliefs and Protective Behaviors Among United States Internet Users. \emph{SOUPS 2015}. Acceptance Rate 24\%.

\textbf{Rader, E.} and Gray, R.\footref{gdfn} (2015). Understanding User Beliefs About Algorithmic Curation in the Facebook News Feed. \emph{CHI 2015}. Acceptance Rate 25\%. doi: 10.1145/2702123.2702174

\textbf{Rader, E.} (2014). Awareness of Behavioral Tracking and Information Privacy Concern in Facebook and Google. \emph{SOUPS 2014}. Acceptance Rate 27\%.

Wash, R., \textbf{Rader, E.}, Vaniea, K. and Rizor, M.\footref{ugfn} (2014). Out of the Loop: How Automated Software Updates Cause Unintended Security Consequences. \emph{SOUPS 2014}. Acceptance Rate 27\%.

Vaniea, K., \textbf{Rader, E.}, and Wash, R. (2014). Betrayed by updates: how negative experiences affect future security. \emph{CHI 2014}. Acceptance Rate 23\%. doi: 10.1145/2556288.2557275

\textbf{Rader, E.}, Velasquez, A.\footref{gdfn}, Hales, K., and Kwok, H.\footref{ugfn} (2012). The Gap Between Producer Intentions and Consumer Behavior in Social Media. \emph{GROUP 2012}. Acceptance rate 37\%. doi: 10.1145/2389176.2389213

\textbf{Rader, E.}, Wash, R., and Brooks, B.\footref{gdfn} (2012). Stories as Informal Lessons About Security. \emph{SOUPS 2012}. Acceptance rate 21\%. doi: 10.1145/2335356.2335364

Wash, R. and \textbf{Rader, E.} (2011). Influencing Mental Models of Security: A Research Agenda. \emph{NSPW 2011}. doi: 10.1145/2073276.2073283

\textbf{Rader, E.\footref{rolefn}}, Echelbarger, M. and Cassell, J. (2011). Brick by brick: Iterating interventions to bridge the achievement gap with virtual peers. \emph{CHI 2011}, 2971-2974. Acceptance rate 27\%. doi: 10.1145/1978942.1979382

\textbf{Rader, E.} (2010). The Effect of Audience Design on Labeling, Organizing, and Finding Shared Files. \emph{CHI 2010}, 777-786. Acceptance rate 22\%. doi: 10.1145/1753326.1753440

\textbf{Rader, E.\footref{gdfn}} (2009). Yours, Mine, and (Not) Ours: Social Influences on Group Information Repositories. \emph{CHI 2009}, 2095-2098. Best Paper Honorable Mention (top 5\% of papers). Acceptance rate 25\%. doi: 10.1145/1518701.1519019

\textbf{Rader, E}.\footref{gdfn} and Wash, R.\footref{gdfn} (2008). Influences on Tag Choices in del.icio.us. \emph{CSCW 2008}, 239-248. Acceptance rate 23\%. doi: 10.1145/1460563.1460601

Wash, R.\footref{gdfn} and \textbf{Rader, E}.\footref{gdfn} (2007). Public Bookmarks and Private Benefits: An Analysis of Incentives in Social Computing. \emph{ASIS\&T Annual Meeting 2007.}

Batcheller, A.\footref{gdfn} L., Hilligoss, B.\footref{gdfn}, Nam, K.\footref{gdfn}, \textbf{Rader, E.\footref{rolefn}}\footref{gdfn}, Rey-Babarro, M.\footref{gdfn}, \& Zhou, X.\footref{gdfn} (2007). Testing the Technology: Playing Games with Video Conferencing. \emph{CHI 2007}, 849-852. Acceptance rate 25\%. doi: 10.1145/1240624.1240751

\textbf{Patrick (Rader), E.}\footref{gdfn}, Cosgrove, D.\footref{gdfn}, Slavkovic, A.\footref{gdfn}, Rode, J.A.\footref{gdfn}, Verratti, T.\footref{gdfn} and Chiselko, G.\footref{gdfn} (2000). Using a large projection screen as an alternative to head-mounted displays for virtual environments. \emph{CHI 2000}, 478-485. Acceptance rate 21\%. doi: 10.1145/332040.332479


\newpage
%----------------------------------------------------------------------
% CONFERENCE AND WORKSHOP PAPERS
\section{Conference and Workshop Papers}

Vaniea, K., \textbf{Rader, E.}, and Wash, R. (2014). Mental Models of Software Updates. \emph{International Communication Association},  Seattle, WA.

Wash, R. and \textbf{Rader, E.} (2010). Using Economic Modeling to Predict User Behavior. \emph{CHI 2010 Workshop: Models, theories and methods of studying online behavior}, Atlanta, GA.

\textbf{Rader, E.} (2008). Group Information Repositories as Social Systems. \emph{4th Annual SIG SI Social Informatics Research Symposium}, Columbus, OH.

Wash, R. and \textbf{Rader, E.} (2008). Understanding del.icio.us Tag Choice Using Simulations. \emph{iSchools Conference 2008}, Los Angeles, CA.

Hofer, E., \textbf{Rader, E.}, and Finholt, T. (2005). Toward supporting virtual collocation. \emph{WACE 2005 (Workshop on Advanced Collaborative Environments)}, Seattle, WA.

\textbf{Rader, E.} (2002). Small Group Communication over the Access Grid: A Case Study. \emph{Proceedings of the 2002 Access Grid Retreat}, San Diego, CA.

\textbf{Rader, E.} (2001). Barriers to Collaboration: User-Centered Research and the Access Grid. \emph{Proceedings of the 2001 Access Grid Retreat}, Argonne National Labs.



%----------------------------------------------------------------------
% POSTERS
\section{Extended Abstracts and Posters}

Hautea, S., Munasinghe, A. and Rader, E. (2020). `That's Not Me': Surprising Algorithmic Inferences. To appear in the Extended Abstracts of the 2020 CHI Conference on Human Factors in Computing Systems. Acceptance rate 41.8\%.

Nthala, N. and Rader, E. (2020). Towards a Conceptual Model for Provoking Privacy Speculation. To appear in the Extended Abstracts of the 2020 CHI Conference on Human Factors in Computing Systems. Acceptance rate 41.8\%.

Cotter, K., Cho, J., and Rader, E. (2017). Explaining the News Feed Algorithm: An Analysis of the ``News Feed FYI'' Blog. Extended Abstracts of the 2017 CHI Conference on Human Factors in Computing Systems. Acceptance rate 38.7\%.

Rader, E. and Wash, R. (2017). Influencing Mental Models of Security. National Science Foundation Secure and Trustworthy Cyberspace PI Meeting.

Wash, R. and Rader, E. (2016). Human Interdependencies in Security Systems. \emph{CCC Visioning Workshop on Grand Challenges in Sociotechnical Cybersecurity}, College Park, MD.

Berman, R., Wash, R. and Rader, E. (2015). Predicting Password Practices. Mid-Michigan Symposium for Undergraduate Research Experiences (MiD-SURE).

Finch, J. and Rader, E. (2015). Differentiating the Effects of Social Influence and Homophily on Social Networks. Mid-Michigan Symposium for Undergraduate Research Experiences (MiD-SURE).

Pinto Santos, R.P., Wash, R., and Rader, E. (2015). Cleaning Computer Security Data. Mid-Michigan Symposium for Undergraduate Research Experiences (MiD-SURE).

Rosemurgy, P. and Rader, E. (2015). Modeling the Diffusion of Shared Votes on Facebook. Mid-Michigan Symposium for Undergraduate Research Experiences (MiD-SURE).

Hoban, K., Rader, E., Wash, R., and Vaniea, K. (2014). Computer Security Information in Stories, News Articles, and Education Documents. \emph{SOUPS 2014.} \textbf{Distinguished Poster Award Winner.}

Jung, Y. and Rader, E. (2014). Transitive Privacy Concern in Social Networks. \emph{SOUPS 2014.}

Baker, C. and Rader, E. (2014). Algorithmic Curation: Understanding the Content Network. Mid-Michigan Symposium for Undergraduate Research Experiences (MiD-SURE).

Bisht, S., Vaniea, K., Wash, R., and Rader, E. (2014). What Does Computer Security Cost You? Mid-Michigan Symposium for Undergraduate Research Experiences (MiD-SURE).

Rose, P. and Rader, E. (2014). Online Information Privacy: Understanding User Concern and Behavior. MSU University Undergraduate Research and Arts Forum (UURAF).

Hoban, K., Rader, E., Vaniea, K., and Wash, R. (2014). Computer Security: Stories, Articles, and Education Documents. MSU University Undergraduate Research and Arts Forum (UURAF).

Elhazzat, J., Hasselbeck, T., Vaniea, K., Rader, E., and Wash, R. (2014). Reconstructing Digital Events. MSU University Undergraduate Research and Arts Forum (UURAF).

Rizor, M., Vaniea, K., Rader, E., Wash, R. (2013). Out of the Loop: How Automated Software Updates Cause Unintended Security Consequences. MSU Cyberinfrastructure Days 2013. \textbf{Winner of the 2nd place prize for Best Poster.}

Heldt, R., Vaniea, K., Wash, R., and Rader, E. (2013). The difficulties of Tracking Users' Online Passwords. Mid-Michigan Symposium for Undergraduate Research Experiences (MiD-SURE). 

Hinck, A., Wash, R., Vaniea, K., Rader, E. (2013). What do YOU Think is a Virus? Mid-Michigan Symposium for Undergraduate Research Experiences (MiD-SURE).

Hoban, K., Girouard, Z., Rader, E., Vaniea, K., and Wash, R. (2013). Computer and Information Security: Educating the User. Mid-Michigan Symposium for Undergraduate Research Experiences (MiD-SURE).

Saxton, N., Vaniea, K., Wash, R., Rader, E. (2013). Tracking Browsing Context in Internet Security Behaviors Research. Mid-Michigan Symposium for Undergraduate Research Experiences (MiD-SURE).

Olsen, T., Saxton, N., Vaniea, K., Wash, R., and Rader, E. (2012). Do People Behave Securely Online? MSU University Undergraduate Research and Arts Forum (UURAF).

Rizor, M., Wash, R., Vaniea, K., and Rader, E. (2012). User Understanding of Security Software Updates: An Interdisciplinary Approach. MSU University Undergraduate Research and Arts Forum (UURAF).

Zemanek, N., Velasquez, A., Rader, E. and Wash, R. (2012). Content Analysis of Security Education Materials Created for User Consumption. MSU Summer Undergraduate Research Forum (SURF).
  
Rader, E. (2007). Just email it to me! Why things get lost in shared file repositories. \emph{GROUP 2007}, Nov. 4-7, 2007, Sanibel Island, FL.

Rader, E. and Wash, R. (2006). Tagging with del.icio.us: Social or Selfish? \emph{CSCW 2006}, Nov. 4-8 2006, Banff, Alberta, CA.
 

%----------------------------------------------------------------------
% PRESENTATIONS
\section{External Presentations}

Rader, E. (2019). Folk Theories of Security and Privacy. \href{https://www.rsaconference.com/industry-topics/presentation/security-privacy-and-human-behavior}{RSAConference: Security, Privacy and Human Behavior (pre-conference seminar)}. March 4, 2019.

Rader, E. (2019). Folk Theories of Security and Privacy. \href{https://www.rsaconference.com/industry-topics/presentation/security-privacy-and-human-behavior}{RSAConference: Hacking the Human: Special Edition (full conference session)}. March 5, 2019.

Rader, E. (2018). \href{https://calendar.colorado.edu/event/info_seminar_emilee_rader_implications_of_beliefs_about_derived_personal_data}{Implications of Beliefs about Derived Personal Data for Negotiating Digital} \href{https://calendar.colorado.edu/event/info_seminar_emilee_rader_implications_of_beliefs_about_derived_personal_data}{Privacy Norms}. Department of Information Science INFO Seminar, \emph{University of Colorado, Boulder}, November 14, 2018.

Rader, E. (2018). \href{https://spice.sice.indiana.edu/2018/11/02/spice-talk-series-features-professor-emilee-rader}{Implications of Beliefs about Derived Personal Data for Negotiating Digital} \href{https://spice.sice.indiana.edu/2018/11/02/spice-talk-series-features-professor-emilee-rader}{Privacy Norms}. Center for Security and Privacy in Informatics (SPICE), \emph{Indiana University}, October 31, 2018.

Rader, E. (2018). Inferring User Behaviors from Log Data for Understanding Computer Security Decisions. \href{https://icosbigdatacamp.github.io/2018-summer-camp/}{ICOS Big Data Summer Camp}, \emph{University of Michigan}, May 14, 2018.  

Rader, E. (2018). Conceptualizing Derived Data as a Social Dilemma. College of Information Studies, \emph{University of Maryland}, April 9, 2018.

Rader, E. (2018). Conceptualizing Digital Privacy as a Social Dilemma. College of Information Sciences and Technology, \emph{The Pennsylvania State University}, March 12, 2018.

Rader, E. (2018). Conceptualizing Digital Privacy as a Social Dilemma. School of Computing and Information, \emph{University of Pittsburgh}, February 26, 2018.

Rader, E. (2018). Conceptualizing Digital Privacy as a Social Dilemma. School of Information Sciences, \emph{University of Illinois}, February 21, 2018

Rader, E. (2017). \href{https://engineering.wustl.edu/Events/Pages/CSE-Colloquia-Series-Emilee-Rader.aspx}{Conceptualizing Digital Privacy as a Social Dilemma}. CSE Colloquia Series at \emph{Washington University in St. Louis}, October 20, 2017.

Rader, E. (2017). Conceptualizing Derived Data as a Common Pool Resource. Security in Human Behavior Workshop, \emph{Cambridge University}, May 26, 2017.

Rader, E. (2016). Examining Derived Data as a Common Pool Resource. CSCW @ Scale Workshop at the School of Information, \emph{University of Michigan}, May 6, 2016.

Rader, E. (2016). Stories, News, and Advice: Patterns in Informal Sources of Security Information. College of Information Sciences and Technology, \emph{The Pennsylvania State University}, March 14, 2016.

Rader, E. (2016). Accidental Transparency: How the Facebook News Feed Algorithm Affects User Beliefs and Behavior. School of Information Studies, \emph{Syracuse University}, March 10, 2016.

Rader, E. (2015). A Filter Bubble For Relationships, Algorithmic Transparency Workshop, Tow Center for Digital Journalism at \emph{Columbia University}, March 27, 2015.

Rader, E. (2011). Understanding Infrastructure for Social Information Sharing, Department of Telecommunication, Information Studies and Media, \emph{Michigan State University}, April 15, 2011.

Rader, E. (2010). How ``Social'' is the Social Web? Department of Communication Studies, \emph{Northwestern University}, January 22, 2010.

Rader, E. (2010). How ``Social'' is the Social Web? Department of Information Systems, \emph{University of Maryland Baltimore County}, January 14, 2010.

Rader, E. (2010). Exploring the ``Social'' Nature of Social Media in Organizations. \emph{NCA 2010 Panel: Social Media and the Communication of Knowledge in the Workplace}, San Francisco, CA.

Rader, E. (2009). Just Email It to Me! Why Things Get Lost in Group Information Repositories. School of Communication and Information Studies, \emph{Rutgers University}, February 6, 2009.

Rader, E. (2007). What Did You Call It? How Document Labeling and Organizing Affect Finding in Shared Repositories. \emph{Johns Hopkins University Applied Physics Laboratory}, November 8, 2007

Rader, E. (2007). What did you call it again? Language Use in Group Information Management Systems. \emph{GROUP 2007 Doctoral Colloquium}, Sanibel Island, FL.

Rader, E. and Johnston, E. (2005). Cooperation and Competition in Video Games: Implications for Collaborative Systems. \emph{Connections 2005}, Montreal, Quebec, Canada.

Patrick, E. (2001). Access Grid Node User Studies: Distributed Video Mediated Communication. \emph{Motorola Labs 2001 Real Technology Series}, Schaumburg, IL.

Patrick, E. (2001). User Centered Evaluation of the Access Grid. \emph{Access Grid Human Factors Workshop}, Argonne National Labs.

Patrick, E. (2001). User Centered Research and the Access Grid: Requirements for Collaboration. \emph{Access Grid Retreat 2001}, Argonne National Labs.

Patrick, E. (2001). The Access Grid: Beyond Videoconferencing. \emph{Chicago chapter of ACM SIG-CHI}, Chicago IL.


%----------------------------------------------------------------------
% BOOK CHAPTERS
\section{Book Chapters}

Wash, R. and \textbf{Rader, E.} (2012). Folk Models of Home Computer Security. Markus Jacobsson (Ed.). \emph{The Death of the Internet,} Wiley; ISBN 978-1118062418

Birnholtz, J., \textbf{Rader, E.}, Horn, D.B., and Finholt, T. (2009). Enabling Remote Participation in Research. B. Whitworth and A. de Moor (Eds.) \emph{Handbook of Research on Socio-Technical Design and Social Networking Systems}. Information Science Reference; ISBN 978-1605662640


%----------------------------------------------------------------------
% TECHNICAL REPORTS
\section{Technical Reports}

Metcalf, C. and \textbf{Patrick, E.} Sharing Photos: Sending, Receiving, and Reviewing Personal Photographs with Family and Friends. \emph{Motorola Labs Technical Report}, February 2003, Schaumburg IL.

\textbf{Patrick, E.} and Metcalf, C. Mediated Communication Between Extended Family and Friends: A Case Study. \emph{Motorola Labs Technical Report}, September 9, 2001, Schaumburg IL.

\textbf{Patrick, E.} and Marturano, L. GTSS Information Architecture User Study Report. \emph{Motorola Labs Technical Report}, June 2001, Schaumburg IL.

\textbf{Patrick, E.} Motorola Labs Website Usability Report. \emph{Motorola Labs Technical Report}, January 2000, Schaumburg IL.

\textbf{Patrick, E.} Motorola ``CoPilot'' Usability Evaluation. \emph{Motorola Labs Technical Report}, November 1999, Schaumburg IL.


%----------------------------------------------------------------------
% INTERNAL PRESENTATIONS 
\section{Internal Presentations}
\begin{tabbing}

2019 \hspace{0.3in}\=  \emph{RCR Session: Protection of Human Subjects}. Topic: Informed consent issues when \\ 
\> \hspace{0.5cm} doing online/digital research. October 14, 2019. \\\\

2018 \hspace{0.3in}\=  \emph{Quantitative Research Design, ADV 975} Topic: Inferring User Behaviors from \\ 
\> \hspace{0.5cm} Log Data for Understanding Computer Security Decisions. November 19, 2018. \\\\

2017 \hspace{0.3in}\=  \emph{Catalyst Talks: Visioning the Future of Doctoral Training in ComArtSci} \\ 
\> \hspace{0.5cm} Topic: Pasteur's Quadrant. November 10, 2017. \\\\

2016 \hspace{0.3in}\=  \emph{Methods for Understanding Users, MI 220} Topic: Contextual Inquiry Interpretation \\
\> \hspace{0.5cm} Sessions and Affinity Diagrams. Nov. 2 and Nov. 7, 2016. \\
\> \emph{MIS PhD 2016 Research Seminar Series} Topic: How the Facebook News Feed Algorithm \\
\> \hspace{0.5cm} Affects User Beliefs and Behavior. February 4, 2016.\\\\\\

2015 \hspace{0.3in}\=  \emph{Computer-Assisted Reporting, Journalism 407} Topic: Algorithmic curation and the \\
\> \hspace{0.5cm} news. October 13, 2015. \\
\> \emph{Sharper Focus, Wider Lens, MSU Honors College Panel} Topic: A Filter Bubble \\
\> \hspace{0.5cm} For Relationships, March 30, 2015.\\
\> \emph{Social Media News \& Information, Journalism 821} Topic: Algorithmic curation and the \\
\> \hspace{0.5cm} news. March 16, 2015. \\\\

2014 \hspace{0.3in}\=  \emph{MIS Proseminar} Topic: Human Computer Interaction. September 5, 2014.\\\\

2013 \hspace{0.3in}\=  \emph{MSU Cyberinfrastructure Days Panel} Topic: Research-Enabling Tools and Technologies:\\
\> \hspace{0.5cm} Challenges and Experiences from a MSU PI Perspective. October 25, 2013.\\ 
\> \emph{Lab Environmental Reporting, Journalism 472} Topic: Face Recognition. October 3, 2013. \\
\> \emph{MIS Program Job Market Panel} September 27, 2013. \\
\> \emph{Research Data Management Cafe} Topic: Data Management in Interdisciplinary \\
\> \hspace{0.5cm} Projects. April 29, 2013. \\\\ % newpage

2012 \hspace{0.3in}\=  \emph{MIS Proseminar} Topic: Information Science. March 16, 2012.\\
\> \emph{Media Impacts on Society, TC401.} Topic: Incentives and constraints in online \\
\> \hspace{0.5cm} communities. MSU Guest Lecture. November 15, 2012. \\\\

2011 \hspace{0.3in}\= \emph{Applied Research Methods, (MTS 525)} NU Guest Lecture, Feb. 28, 2011. \\
\> \emph{MSU Comm Arts Brownbag Series}, Oct. 27, 2011. \\
\> \emph{MIS Program Job Market Panel}, Sept. 30, 2011. \\\\

2010 \hspace{0.3in}\= \emph{NU SoC Century Scholars Invited Talk}, Nov. 11, 2010. \\
\> \emph{MTS Graduate Seminar (MTS 501)}. NU Guest Lecture. \\
\> \emph{Applied Research Methods, (MTS 525)} NU Guest Lecture. \\\\

2009 \hspace{0.3in}\= \emph{Doctoral Foundations (SI 701)}. SI Guest Lecture. \\
\> \emph{SI Academic Job Search Panel}, Apr. 23, 2010. \\\\

2008 \hspace{0.3in}\= \emph{Human Interaction in Information Retrieval (SI 531)}. SI Guest Lecture. \\
\> \emph{Research Methods (SI 840)}. SI Guest Lecture. \\\\

2007 \hspace{0.3in}\= \emph{CSCW (SI 689)}. SI Guest Lecture. \\
\> \emph{Public Goods (SI 686)}. SI Guest Lecture. \\
\> \emph{Research Methods (SI 840)}. SI Guest Lecture. \\
\> \emph{Organization of Information Resources (SI 666)}. SI Guest Lecture. \\\\

2006 \hspace{0.3in}\= \emph{SI CREW Seminar Invited Talk}, Apr. 10, 2006. \\
\end{tabbing}



%--------------------------------------------------------------------------
% AWARDS AND HONORS
%\section{Awards and Honors}
%
%\begin{tabbing}
%2011 \hspace{0.3in}\= Accepted the endowed AT\&T Scholar position, in the Media and Information \\
%\> \hspace{0.5cm} Department at Michigan State University. \\\\
%
%2010 \hspace{0.3in}\= Selected to participate in the CCC/CRA \href{http://icie.cs.byu.edu/CCCWorkshops/Ultra-large-scale.html}{Ultra-large-scale Interaction Workshop} \\\\
%
%2009 \> Selected to participate in the \href{http://sociotech.net/}{CSST '09 Summer Research Institute} \\ 
%\> CHI '09 paper received Best Paper Honorable Mention (top 5\% of submissions)
%\end{tabbing}


%--------------------------------------------------------------------------
% UNIVERSITY SERVICE
\section{University Service}

\emph{Michigan State University} \\
\vspace{1pt}
\begin{tabbing}
2019 \hspace{0.3in}\= Director of Doctoral Studies, Media and Information Dept., and \\
\> \hspace{0.5cm} representative on the IM PhD Executive Committee (Spring) \\
\> Leader of the MI Dept. Human Centered Technologies working group (Spring) \\\\

2018 \hspace{0.3in}\= Director of Doctoral Studies, Media and Information Dept., and \\
\> \hspace{0.5cm} representative on the IM PhD Executive Committee \\
\> Leader of the MI Dept. Human Centered Technologies working group \\\\

2017 \hspace{0.3in}\= Director of Doctoral Studies, Media and Information Dept., and \\
\> \hspace{0.5cm} representative on the IM PhD Executive Committee \\\\

2016 \hspace{0.3in}\= Director of Doctoral Studies, Media and Information Dept., and \\
\> \hspace{0.5cm} representative on the IM PhD Executive Committee (Fall) \\
\> Co-Organized IM PhD Research Symposium (Fall) \\ % Fall
\> Member of the Media and Information Dept. PhD Committee (Spring) \\ 
\> Member of the Media and Information Dept. Work Load Task Force (Spring) \\\\

2015 \hspace{0.3in}\= Member of the Media and Information Dept. PhD Committee (Fall) \\ % Fall
\> Co-Organized MIS PhD Research Symposium (Fall) \\ % Fall
\> Served on the MIS PhD Program Review Committee (Spring) \\ % Spring
\> Reviewer for CAS summer research proposals (Spring) \\\\ % Spring

2014 \hspace{0.3in}\=  Panelist for MSU Women's Alliance session on ``Working Women'' \\ % Spring
\> Co-Organized MIS PhD Research Symposium \\ % Spring
\> Presented a Responsible Conduct of Research session on ``Management of Data'' \\\\ % March 21, 2014

2013 \hspace{0.3in}\=  Director of Graduate Studies, TISM \\
\> Co-Organized MIS PhD Research Symposium \\\\ 

2012 \>  Media and Information Studies PhD Committee \\
\> Co-Organized First Annual MIS PhD Research Symposium, and served as judge \\ % Spring
\> Served as a mentor for a \href{https://grad.msu.edu/srop}{Summer Research Opportunities (SROP)} student \\ % Summer
\> Served on a university-wide IT services advisory committee \\ % Spring
\end{tabbing}

\emph{Northwestern University, Communication Studies Department} \\
\vspace{1pt}
\begin{tabbing}
2011 \hspace{0.3in}\=  Technology and Social Behavior PhD Admissions committee \\\\
2010 \> Organizer of TSB Seminar Series (with Darren Gergle) \\
\> Chair's Advisory Committee, Department of Communication Studies \\
\end{tabbing}

\emph{University of Michigan, School of Information} \\
\vspace{1pt}
\begin{tabbing}
2006  \hspace{0.3in}\= PhD Admissions Committee \\\\
2005 \> President of the Doctoral Student Organization \\
\> Doctoral Committee
\end{tabbing}
 

%--------------------------------------------------------------------------
% PROFESSIONAL SERVICE
\section{Professional Service}

\emph{Leadership} \\ 
\vspace{1pt}
Co-Chair of the Privacy \& Security Subcommittee of the ACM Conference on \\
\hspace{0.5cm} Computer-Human Interaction, 2019-present \\
\vspace{1pt}
Organizer of the Workshop on Trustworthy Algorithmic Decision-Making, Fall 2017 \\
\hspace{0.5cm} Workshop Dates: December 4-5, 2017; Website: \href{http://trustworthy-algorithms.org/}{http://trustworthy-algorithms.org/}

\emph{Member of Conference Program Committees} \\
\vspace{1pt}
ACM Conference on Computer-Human Interaction (CHI) 2010, 2012, 2014 \\
ACM Conference on Computer-Supported Cooperative Work and Social Computing \\
\hspace{0.5cm} (CSCW) 2013, 2015, 2017-18 \\
SOUPS (Symposium on Usable Privacy and Security) 2013, 2015-2017, 2019

\emph{Reviewer (Journals)} \\
\vspace{1pt}
Computers \& Security \\ 
Digital Journalism \\
Human Computer Interaction Journal \\
IEEE Security \& Privacy \\
International Journal of Human Computer Studies \\
JASIST: Journal of the Association for Information Science and Technology \\
Journal of Web Science \\
Journal of Cybersecurity \\ 
PACM HCI (CSCW) \\
PACM HCI (IMWUT) \\ 
TOCHI: Transactions on Computer Human Interaction


\emph{Reviewer (Conferences)} \\
\vspace{1pt}
CSCW 2008-2018 \\
CHI 2006-2019 \\
ICIS 2019 \\
iConference BIAS2018 workshop \\
WebSci 2016 \\
IUI 2015, 2019 \\
Graphics Interface 2015 (HCI Track) \\
UIST 2014 \\
WWW 2012 \\
iConference 2012 \\
ICWSM 2011 \\
GROUP 2009

\emph{Invited Panelist (Grant Proposal Reviewer)} \\
\vspace{1pt}
Israel Science Foundation, 2016 \\
National Science Foundation (NSF), 2011-2013, 2016, 2017, 2019 \\
Austrian Research Promotion Agency (FFG), 2011


%--------------------------------------------------------------------------
% TEACHING
\section{Teaching}

\emph{Instructor of Record} \\
\vspace{1pt}
\begin{tabbing}
Digital Footprints: Privacy Online (undergraduate) \\ 
\hspace{0.5cm} Spring 2016, 2017 (as MI 239, required course) \\
\hspace{0.5cm} Spring 2012; Fall 2012, 2013 (as TC 291, special topic course) \\ 
\hspace{0.5cm} Winter 2011 (Northwestern Comm Studies 395) \\\\

Reasoning with Data (undergraduate) \\
\hspace{0.5cm} Spring 2017-2019; Fall 2019 (as MI 320, required course for the \href{http://mi.msu.edu/undergraduate-studies/majors/}{CHCT focus area}) \\\\

Human Computer Interaction (master's, phd) \\
\hspace{0.5cm} Spring 2015 (CAS 992, PhD special topic course) \\
\hspace{0.5cm} Spring 2013-2016 (as MI 845, required course) \\
\hspace{0.5cm} Spring 2012 (as TC 891, special topic course) \\

Methods for Computational Communication (phd) \\
\hspace{0.5cm} Spring 2019 (CAS 992) \\\\

Online Communities (undergraduate) \\
\hspace{0.5cm} Spring 2011 (Northwestern Comm Studies 378) \\\\

Perspectives on Social Media Research (phd)\\
\hspace{0.5cm} Spring 2011 (Northwestern Comm Studies 525) \\\\

\emph{Independent Study Projects Supervised} \\
\vspace{1pt}
Maddie Heise (MI undergraduate): Spring 2019 \\
Kamari Morse (MI undergraduate): Spring 2018 \\
Christina Lang (MI undergraduate): Fall 2017 \\
Janghee Cho (Media \& Information Master's): Spring 2017 \\
Rebecca Gray (MIS PhD): Spring 2013 \\
Yumi Jung (MIS PhD): Fall 2012 \\\\

\emph{Co-Instructor} \\
\vspace{1pt}
Doctoral Development Seminar (University of Michigan, School of Information) \\
\hspace{0.5cm} with Cory Knobel and Yong-Mi Kim, Fall and Winter 2008 \\\\

\emph{Graduate Student Instructor (TA)} \\
\vspace{1pt}
Fundamentals of Human Behavior. Winter 2009 (UMSI 688) \\
Computer-Supported Cooperative Work. Fall 2008 (UMSI 689) \\ 
Networked Computing: Storage, Communication, and Processing. Fall 2007 (UMSI 502) \\
Contextual Inquiry and Project Management. Fall 2006 (UMSI 501) \\
Use of Information. Fall 2005 (UMSI 501) 
\end{tabbing}



%--------------------------------------------------------------------------
% ADVISING


\section{Students/Postdocs Supervised}

\vspace{-5pt}
\footnotesize
\emph{[All students and postdocs are from Michigan State University unless otherwise noted.]\\}
\normalsize
\vspace{5pt}

\emph{Postdoctoral Researchers and Visiting Scholars} \\
\vspace{1pt}
Norbert Nthala, 2019-present \\
Jo\~ao Marcelo Ferraz (Universidade Federal de Pernambuco, Brazil), 2019-present \\
Kami Vaniea, 2012-2014

\emph{PhD Advisor} \\
\vspace{1pt}
Samantha Hautea, 2019-present \\
Yumi Jung, 2012-2018 (graduated Spring 2018) \\
Alina Lungeanu (Northwestern), 2010-2011

\emph{Committee Member} \\
\vspace{1pt}
Chris Fennell (comps) \\
Kelley Cotter (guidance) \\
Janine Slaker (prelim) \\
Soo Jeong Hong (prelim, graduated 2018) \\
Jacob Solomon (dissertation, graduated 2015)

\emph{Master's Students Advised} \\
\vspace{1pt}
Carl Anderson, Media and Information. 2013-2016 (as Chair) \\
Janghee Cho, Media and Information. 2016-2018 (as Advisor and Chair) \\
Xin Ha, Media and Information. 2015-2017 (as Advisor) \\
Chris Hamrick, Media and Information. 2011 (as Committee Member) \\
Bill Morgan, Media and Information. 2012 (as Committee Member) \\
Alison Virag-McCann, Media and Information. 2012-2013 (as Committee Member) \\
Qin Zhang, Media and Information. 2015 (as Advisor)

\emph{Graduate Student Research Assistants} \\
\vspace{1pt}
Irene Kim, Business Analytics Master's, Spring 2019-present \\
Brandon Brooks, MIS PhD. 2011-2012 \\
Shih-Chi Steven Chen, EECS MS, Northwestern. 2010-2011 \\
Janghee Cho, MI MA. 2016-2017 \\
Kelley Cotter, MIS PhD. 2016-2017 \\
Chris Fennell, MIS PhD. Summer 2016 \\
Joe Freedman, CSE Masters. 2018-Spring 2019 \\
Rebecca Gray, MIS PhD. 2013-2014 \\
Yumi Jung, MIS PhD. 2012-2013 \\
Laeeq Khan, MIS PhD. 2012-2013 \\
Andrew Osentoski, Criminal Justice MA. 2015 \\
Chankyung Pak, MIS PhD. 2015-2016 \\
Janine Slaker, MIS PhD. 2015-2016 \\
Jacob Solomon, MIS PhD. 2011-2012 \\
Alcides Velasquez, MIS PhD. 2011-2012

%\emph{Present Undergraduate Research Assistants} \\
%\vspace{1pt}

\emph{Former Undergraduate Research Assistants} \\
\vspace{1pt}
Howard Akumiah, Fall 2012-Spring 2013 \\
Pierre Arellano (Washington State University), \href{http://grad.msu.edu/srop/}{MSU SROP Program} 2012 \\
Cody Baker (University of Evansville), Summer REU 2014 \\
Ruth Berman (Macalester College), Summer REU 2015 \\
Shiwani Bisht (Cornell University), Summer REU 2014 \\
Nina Capuzzi, Fall 2016-Spring 2017 \\
Stephen Cauthen, Fall 2018-Spring 2019 \\
Jallal Elhazzat, Spring 2014 \\
James Finch, Summer REU 2015 \\
Nicholas Gilreath, Fall 2016-Spring 2017 \\
Zack Girouard, Fall 2012-Summer 2013 \\
Leanarda Gregordi, Fall 2012 \\
Tim Hasselbeck, Spring 2014 \\
Madison Heise, Fall 2018-Spring 2019 \\
Raymond Heldt, Summer REU 2013 \\
Alexandra Hinck (Beloit College), Summer REU 2013 \\
Katie Hoban, Summer REU 2013 and 2014; and Fall 2013-Fall 2014 \\
Meghan Huynh, Fall 2014-Spring 2016 \\
Nathan Klein (Oberlin College), Summer REU 2014 \\
Kyle Kulesza, Fall 2013-Spring 2014 \\
Helen Kwok (Northwestern University), 2011 \\
Lauren McKown, Spring-Summer 2012 \\
Manotej Meka, Spring 2016 \\
Simone Merendi, Spring 2016 \\
Anjali Munasinghe, Fall 2017-Fall 2019 \\
Robert Novak, Fall 2016-Spring 2017 \\
Cindy Ochoa, Fall 2016-Spring 2018. \\
Tyler Olsen, Fall 2012-Spring 2013 \\
Ruchira Ramani, Fall 2013 \\
Michelle Rizor, Fall 2012-Spring 2014 \\
Paul Rose, 2012-2015 \\
Paul Rosemurgy, Fall 2014-Spring 2015; and Summer REU 2015 \\
Scott Rucinski, Fall 2013-Spring 2014 \\
Robert Plant Pinto Santos (Universidade Federal do Cear\'{a}, Brazil), Summer REU 2015 \\
Nick Saxton, Summer REU 2013 and Fall 2012-Fall 2013 \\
Kimberly Setili, Spring 2012 and Summer REU 2012 \\
Alison Thierbach, Summer REU 2012 \\
Elaine Wang (Northwestern), 2011 \\
Zac Wellmer, Fall 2015 \\
Nate Zemanek, Summer REU 2012 \\


\end{document}






